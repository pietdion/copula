\documentclass[useAMS,usenatbib,referee]{article}
\usepackage{latexsym}
\usepackage{rotate}
\usepackage{natbib}
\usepackage{graphics}
\usepackage{threeparttable}
\usepackage{comment}
\usepackage{eurosym}


%\bibliographystyle{unsrt}
\bibliographystyle{chicago}
%\biboptions{super}

\newcommand{\ssection}[1]{%
  \section[#1]{\textbf{\uppercase{#1}}}}
\newcommand{\ssubsection}[1]{%
  \subsection[#1]{\normalfont\textbf{#1}}}


%\renewcommand{\labelenumi}{(\roman{enumi})}
%\newcommand{\euro}{\AA}
\newcommand{\eps}{\epsilon}
\newcommand{\var}{{\rm var}}
\newcommand{\cov}{{\rm cov}}
\newcommand{\nid}{{\rm NID}}
\newcommand{\diag}{{\rm diag}}
\newcommand{\E}{{\mathrm E}}
\newcommand{\R}{x}
\newcommand{\U}{{\mathrm U}}
\newcommand{\Ex}{{\cal E}}
\newcommand{\Ax}{{\cal A}}
\newcommand{\Rx}{\mathrm{R}}
\newcommand{\N}{{\cal N}}

\newcommand{\Cx}{{\cal C}}
\newcommand{\cor}{\mathrm{cor}}
\newcommand{\tr}{\mathrm{tr}}
\newcommand{\e}{\mathrm{e}}
\newcommand{\de}{\mathrm{d}}
\newcommand{\p}{\mathrm{P}}
\newcommand{\Ln}{\mathrm{Ln}}
\newcommand{\ra}{\varrho}
\newcommand{\ph}{\phi}
\newcommand{\ps}{s}
\newcommand{\Ph}{\Phi}
\newcommand{\mf}{\tilde f}
\newcommand{\implies}{\qquad\Rightarrow\qquad}
\newcommand{\cd}{\bullet}
\renewcommand{\P}{\mathrm{P}}
\newcommand{\W}{\mathrm{W}}
\newcommand{\Q}{\mathrm{Q}}
\newcommand{\M}{\mathrm{M}}
\newcommand{\Z}{\mathrm{Z}}
\newcommand{\C}{\mathrm{C}}
\newcommand{\B}{\ensuremath\Psi}
\newcommand{\cq}{\ , \qquad}
\newcommand{\mgf}{g}

\newcommand{\betah}{\beta^*}
\newcommand{\ch}{c^*}
\newcommand{\Ch}{C^*}
\renewcommand{\ph}{p^*}
\newcommand{\qh}{q^*}
\newcommand{\rh}{r^*}
\renewcommand{\b}{\Psi}

\renewcommand{\C}{\ensuremath{\mathrm{^oC}\ }}


\newcommand{\nS}{\neg S}
\newcommand{\nT}{\neg T}
\newcommand{\eref}[1]{(\ref{#1})}
\newcommand{\fref}[1]{Figure \ref{#1}}
\newcommand{\sref}[1]{\S\ref{#1}}
\newcommand{\tref}[1]{Table \ref{#1}}
\newcommand{\aref}[1]{Appendix \ref{#1}}



\title{Copula notes}

\author
{ Piet de Jong\emailx{piet.dejong@mq.edu.au} \\
Department of Applied Finance, Macquarie University, NSW 2109, Australia}

\begin{document}



\section*{Copulas: summary and proofs}

Write $F$ as the probability distribution of random vector $x$ and $F_*$ as the corresponding vector of marginal distributions: $F_*(x)$  evaluates the marginal densities of $F$ at the respective components of $x$.  If $h$ is a vector of increasing component maps then $F\circ h$ is the distribution of $h^-(x)$  and  $(F\circ h)_*=F_*\circ h$.

A copula is a probability distribution $C$ on the unit hypercube with uniform marginals: $C_*=I$.  
Define the copula generated by $F$ as $
C_F\equiv F\circ F_*^-$ and the distribution generated by $C$ from $F_*$ as 
$
F_C\equiv C\circ F_*$.  
\begin{enumerate}
\item $(C_F)_*=(F\circ F_*^-)_*= I$ hence $C_F$ is a copula,  and $C_C=C\circ C_*^- =  C$
\item $(F_C)_*=(C\circ F_*)_*= C_*\circ F_*=F_*$ and $F=C_F\circ F_*$
\item $x\sim F_C$ implies $F_*(x)\sim C$ while $u\sim C_F$ implies $F_*^-(u)\sim F$
\item $ F_{C_F} =F\circ F_*^-\circ F_*=F$ and $C_{F_C}\equiv F_C\circ (F_C)_*^-=C\circ F_*\circ  F_*^-=C$ 
\item  {\bf Monotonic transform}: $C_{F\circ h}=(F\circ h) \circ (F\circ h)_*^- = F\circ h\circ h^- \circ F_*^- = C_F$
    \item {\bf Copula density}: $C_F$ has density
    $
   c_F\equiv (f\circ F_*^-)/(\Pi \circ f_*\circ F_*^-)
    $
    where $f$ and $f_*$ are the densities  corresponding to $F$ and $F_*$, respectively.  
    
      \item {\bf Conditional}: $c(u^i|u_i) = c(u^i,u_i)/c(u_i) = c(u)$;  
$
C(u^i|v_i)=\int_0^{u^i} c(v)\de v^i 
$

 \item {\bf Independence}:  If $F=\Pi\circ F_*$  then $C_F=\Pi\circ F_*\circ F_*^-=\Pi$.  Further $f=\Pi\circ f_*$ and $f\circ F_*^-=\Pi\circ f_*\circ F_*^-$ and $c_F=1$.
 
 
 \item  {\bf Perfect dependence}: $x=h(1\eps)$, $\eps\sim G$ is univariate and $h$ as above.    Then 
 $
 P(x\le z)=P\{1\eps\le h^-(z)\} = (\min\circ G_!\circ h^-)(z)
 $ 
 where $G_!$ applies $G$ to each component.    Hence $F=\min \circ\ G_!\circ h^-$,  $F_*=G_!\circ h^- $  and  $C_F=\min $.
 
    \item  {\bf Gaussian copula} is $C_{\Phi}=\Phi\circ \Phi_*^-$ where $\Phi$ is the Gaussian distribution with standard normal marginals.   The meta--Gaussian model is
$
 x=(F_*^-\circ \Phi_*) (\eps)$, $\eps\sim N(0,R)$ where $R$ is the covariance and correlation matrix of $\Phi$.
\item {\bf Meta--gamma} model  is
$
x = (F_*^-\circ\Gamma_*)(\Psi\eps)$, $\eps \sim \Gamma(\mu,\nu)
$
where  $\Psi\equiv\{1,\diag(\psi)\}$, $\psi$ a vector of parameters,  and $\Gamma_*$ the vector of marginal distributions associated with $\Psi\eps$.


\item {\bf Survival copula} is $C\circ 1_-$ with $1_-(u)=1-u$ and $1_-^-=1_-$

\item  {\bf Levy copula} is $L$ defined on the positive orthont such that $L_*=I$

\item $\de F(x)\equiv \de(C\circ F_*)(x)=c(u)(\Pi\circ \de F_*)(x)$.   Top down aversion adjustment is 
$\de F^\downarrow(x)=\phi(u_+)c(u)(\Pi\circ\de F_*)(x)$ and $f^\downarrow(x_i)= \E\{\phi(u_+)|u_i\} f(x_i)$, $u_+$  wrt $C\circ F_*$.
Bottom up is $\de F^\uparrow(x)=(\Pi\circ \phi)(u)c(u)(\Pi\circ \de F_*)(x)$ implying $\de  F^\downarrow(x)/\de F^\uparrow(x)=\phi(u_+)/(\Pi\circ \phi)(u)$, $f^\uparrow(x_i)=\phi(u_i)f(x_i)$ 

\item The correlation matrix of $u\sim C$ is called Spearman's rho.   The means and variances of each component of $u$  are 1/2 and 1/12, respectively.   Hence Spearman's rho correlation matrix is
$$
\frac{1}{1/12}\left\{\E(uu')-\frac{1}{4}11'\right\} = 12\int uu'\de C(u)-3 = 12\int C(u)\de u - 3\ .
$$

\item Kendall's tau is one third of  Spearman's rho calculated with copula $C^2\Pi^{-1}$.   To show this note the total differential of the last expression is 
$$
2C\Pi^{-1}\de C- C^2\Pi^{-2}\de \Pi
$$



To show this note that $\de C(u)=c(u)\de u$ where $\de u$ is the product of the individual differential elements.   Thus $\de C^2=2C\de C=$


 and 


 this suppose $\hat C$ is such that $C=(\hat C\Pi)^{1/2}$.  Then  
$$
\Pi\de C^2(u)=2C(u)\Pi\de C(u)\qquad\Rightarrow\qquad\de C^2(u)=2C(u)\Pi^-(u)\de C(u)\ .
$$
$$
C\de C=(\hat C\Pi)^{1/2}\frac{\hat C^{-1/2}\Pi^{1/2}\de \hat C+\hat C^{1/2}\Pi^{-1/2}\de\Pi}{2} = \frac{\Pi\de\hat C+\hat C\de\Pi}{2}
$$$$= \hat C\de\Pi =C^2\Pi^{-1}\de\Pi\ .
$$
Hence for the bivariate case Kendall's tau is
$$
\tau=4\int C(u,v)\de C(u,v)-1 = 4\int \frac{C^2(u,v)}{uv}\de u \de v-1
= 4\frac{\rho+3}{12} - 1 = \frac{\rho}{3}
$$
where  $\rho$ is Spearman's rho 
calculated with $C^2(u,v)/(uv)$.  Note the marginal distribution is $C^2(u,1)/u=u^2/u=u$ and similarly for $v$ and hence $C^/\Pi$ is a copula.  If $C=\Pi$ then $C^2/\Pi=\Pi$.  Hence $\tau\ne\rho$ even if $C=\Pi$.
\end{enumerate}

\section*{Conormal distributions}

A distribution $N$ is conormal if  $N_*=\Phi_*$ where $\Phi$ is the joint normal with standard normal marginals.
Thus $N_*$ computes standard normal probabilities.  The  conormal defined from  $F$ is
$
N_F \equiv  F\circ P_*^-$ where $P_* =\Phi_*^-\circ F_*$.  The joint $N_F$ is conormal since $(N_F)_*=\Phi_*$.   Further $F=N_F\circ P_*$ and hence any joint can be written as a conormal composed with a function of marginals.  Note
$P_*(x)$ computes z--scores  from the percentiles $F_*(x)$. 

The joint defined by conormal $N$ and $F_*$ is 
$
F_N \equiv N\circ P_*
$.
Note $(F_N)_*= F_*$ and detailed calculations show 
$
N_{F_N} =N
$ and $
F_{N_F} = F
$.
If $F=\Phi$ then $N_F = F=F_N$.  Hence the joint normal with standard normal marginals is the ``identity."

Any joint normal is of the form $\Phi\circ z$ where  $z(x)$ are z--scores.   Since  $
(F\circ z)_* = F_*\circ z$,  the conormal defined by $\Phi\circ z$ is
$$
N_{\Phi\circ z} \equiv  \Phi\circ z\circ (\Phi\circ z)_*^-\circ N_*=\Phi\circ\Phi_*^-\circ N_*=\Phi \cq N_{\Phi\circ z} \circ P_* = \Phi\circ\Phi_*^-\circ F_*\ .
$$
Thus using a conormal defined from a joint normal is equivalent to using a Gaussian copula.
The joint defined by $N$ and  $(\Phi\circ z)_*$ is
$$
(\Phi\circ z)_N = N\circ \Phi_*^-\circ (\Phi\circ z)_* = N\circ z\ , 
$$
the conormal applied to the distribution of  $z(x)$.


\end{document}
